\documentclass[11pt,a4paper]{article}
\input{hwPreliminary.tex}
\usepackage{lipsum}  % Import the lipsum package
\usepackage{algorithm2e,algorithmic}
%\graphicspath{{images/}}
\allowdisplaybreaks
%------------------------------------------------
\begin{document}
\title{Title}
\author{Author Name\thanks{Email: author@example.com} (SID: 123456)}
\date{\today}
\maketitle




%--------------------------
%--------------------------
\section{Question and Solution}

{\color{blue}The question environment is displayed in a blue box, and we have also defined a solution environment:}

\begin{question}[\optional \lvc~---~ A question]
    Your question here...
\end{question}


\begin{sol}
    \lipsum[1]
    \begin{equation}
    \sin(\alpha+\beta)=\sin \alpha \cos \beta + \cos \alpha \sin \beta.
\end{equation}
\end{sol}


\section{Theorem}
{\color{blue} You can also use the following environment to help complete your homework:}
\begin{definition}[Definition] 
Your definition here...
\end{definition}





\begin{lemma}[A Lemma]
    Your lemma here...
\end{lemma}

\begin{remark}[A Remark]
    Your remark here...
    \experiment \take \intuition
\end{remark}

   
\section{Box}
{\color{blue} Four Markdown-style boxes can be used directly:}

\begin{redbox}
    This is a red box.
\end{redbox}

\begin{greenbox}
    This is a green box.
    \end{greenbox}

\begin{graybox}
    This is a gray box.
    \end{graybox}

\begin{bluebox}
    This is a blue box.
  \end{bluebox}

{\color{blue} But you'd better not use them directly to avoid confusion since other environments have used these boxes, try to define your own box (see below).}

{\color{blue}If you don’t like the colors, you can define your own:}

\begin{tex}
\definecolor{mypink}{rgb}{1,0.965,0.965}
\definecolor{blue2}{RGB}{0,47,167}
\end{tex}

{\color{blue}Then update the color in the following code:}

\begin{tex}
\newtcolorbox{bluebox}{
  colback=myblue!3,      % Background color of the box
  colframe=myblue,    % Border color of the box  
  leftrule=4pt,             % Thickness of the left border
  toprule=0pt,              % No top border
  bottomrule=0pt,           % No bottom border
  rightrule=0pt,            % No right border
  arc=1.2mm,                  % Rounded corners
  outer arc=1.5mm,            % Outer border radius
}
\end{tex}

{\color{blue} The same applies to other boxes, and you can define your own in the same way.}

% ------------------------------------------------------------------------------
\section{Figures and Tables}

{\color{blue}Avoid using floating environments (figures and tables) inside the boxes. To include them, use this structure:}

\begin{tex}
    \begin{center}
        \includegraphics[width=0.8\textwidth]{image.jpeg}
        \captionof{figure}{A figure}
    \end{center}
\end{tex}

\begin{tex}
    \begin{center}
        \begin{tabular}{|c|c|c|}
            \hline
            a & b & c \\
            \hline
            a & b & c \\
            \hline
        \end{tabular}
        \captionof{table}{A table}
    \end{center}
\end{tex}

\begin{bluebox}
    \begin{center}
        \includegraphics[width=0.8\textwidth]{image.jpeg}
        \captionof{figure}{A figure}
    \end{center}
    \begin{center}
        \begin{tabular}{|c|c|c|}
            \hline
            a & b & c \\
            \hline
            a & b & c \\
            \hline
        \end{tabular}
        \captionof{table}{A table}
    \end{center}
\end{bluebox}

\section{Code}

{\color{blue}We have defined a code environment for R:}

\begin{R}
> a+1
# [1] 2
\end{R}

{\color{blue}If you're using a different language, modify the following code:}

\begin{tex}
\lstnewenvironment{R}{\lstset{
    language=R,
    basicstyle=\footnotesize\ttfamily,
    numbers=left,
    numberstyle=\tiny\color{black},
    stepnumber=1,
    numbersep=5pt,
    backgroundcolor=\color{mygray},
    showspaces=false,
    showstringspaces=false,
    showtabs=false,
    frame=single,
    rulecolor=\color{black},
    tabsize=4,
    captionpos=b,
    breaklines=true,
    breakatwhitespace=false,
    keywordstyle=\ttfamily\bfseries\color{myblue},
    commentstyle=\ttfamily\bfseries\color{myred},
    stringstyle=\ttfamily\bfseries\color{mygreen}
}}{}
\end{tex}

{\color{blue}I've also defined a \verb|tex| environment for this tutorial. You can remove it if you don't need it.}


\section{Macro}
{\color{blue}This template includes macros shown in Table \ref{table1} and Table \ref{table2}.}

%math operator
\begin{table}[h]
\centering
\setlength{\arrayrulewidth}{1pt} % 设置表格线条粗细
\setlength{\tabcolsep}{10pt}     % 设置列间距
\renewcommand{\arraystretch}{1.5} % 设置表格行高
\begin{tabular}{|>{\centering\arraybackslash}m{3cm}|>{\centering\arraybackslash}m{3cm}||>{\centering\arraybackslash}m{3cm}|>{\centering\arraybackslash}m{3cm}|}
\hline
\textbf{Macro} & \textbf{Symbol} & \textbf{Macro} & \textbf{Symbol} \\
\hline
\texttt{\textbackslash C}  & $ \C $   & \texttt{\textbackslash ninfo\{a\}}    & $\ninfo{a}$ \\
\texttt{\textbackslash Q}  & $ \Q $   & \texttt{\textbackslash ninfc\{a\}}    & $\ninfc{a}$ \\
\texttt{\textbackslash Z}  & $ \Z $   & \texttt{\textbackslash pinfo\{a\}}    & $\pinfo{a}$ \\
\texttt{\textbackslash Rn\{k\}}  & $ \Rn{k} $   & \texttt{\textbackslash  pinfc\{a\}}    & $\pinfc{a}$ \\
\texttt{\textbackslash borel}  & $ \borel $   & \texttt{\textbackslash pa\{a,b,c\}}    & $\pa{a,b,c}$ \\
\texttt{\textbackslash familay}  & $ \family $   & \texttt{\textbackslash br\{a,b,c\}}    & $\br{a,b,c}$ \\
\texttt{\textbackslash oc}  & $ \oc{a,b} $   & \texttt{\textbackslash cbr\{a,b,c\}}    & $\cbr{a,b,c}$ \\
\texttt{\textbackslash co}  & $ \co{a,b} $   & \texttt{\textbackslash inner\{a,b\}}  & $ \inner{a,b} $  \\
 \texttt{\textbackslash norm\{a\}}    & $\norm{a}$ & \texttt{\textbackslash abs\{a\}}    & $\abs{a}$ \\
\texttt{\textbackslash floor}  & $ \floor{a} $   & \texttt{\textbackslash ceil\{a\}}    & $\ceil{a}$ \\



\texttt{\textbackslash dd}  & $ \dd $   & \texttt{\textbackslash dv\{f\}\{x\}\{2\}}    & $\dv{f}{x}{2}$ \\
\texttt{\textbackslash p}  & $ \p $   & \texttt{\textbackslash pdv\{f\}\{x\}\{2\}}    & $\pdv{f}{x}{2}$ \\
  
\texttt{\textbackslash pr}  & $ \pr $   & \texttt{\textbackslash Cov}    & $\Cov$ \\
\texttt{\textbackslash E}   & $\E$   & \texttt{\textbackslash Corr}   & $\Corr$ \\
\texttt{\textbackslash I\{x>1\}} & $\I{x>1}$ & \texttt{\textbackslash inD} & $\inD$ \\
\texttt{\textbackslash inAS}   & $\inAS$ & \texttt{\textbackslash inP} & $\inP$ \\
\texttt{\textbackslash inLp}   & $\inLp$ & \texttt{\textbackslash inMSE} & $\inMSE$ \\
\verb|\simIND|&$\simIND$ & \verb|\indep|& $\indep$\\
\verb|\IID|& $\iid$ & \verb|\simIID| &$\simIID$\\
\verb|\mat{a&b\\c&d}|& $\mat{a&b\\c&d}$ &\verb|\smat{a&b\\c&d}|  &$\smat{a&b\\c&d}$\\
\verb|\bmat{a&b\\c&d}|& $\bmat{a&b\\c&d}$& \verb|\bsmat{a&b\\c&d}| & $\bsmat{a&b\\c&d}$\\
\verb|\pmat{a&b\\c&d}| & $\pmat{a&b\\c&d}$& \verb|\psmat{a&b\\c&d}| &$\psmat{a&b\\c&d}$\\
\verb|\argmin| & $\argmin$ &\verb|\argmax| & $\argmax$ \\
\hline
\end{tabular}
\caption{Macros and Corresponding Symbols for Math Operator}
\label{table1}
\end{table}
%% statistical notation
\begin{table}[h]
\centering
\setlength{\arrayrulewidth}{1pt} 
\setlength{\tabcolsep}{10pt}    
\renewcommand{\arraystretch}{1.5} 
\begin{tabular}{|>{\centering\arraybackslash}m{3cm}|>{\centering\arraybackslash}m{3cm}||>{\centering\arraybackslash}m{3cm}|>{\centering\arraybackslash}m{3cm}|}
\hline
\textbf{Macro} & \textbf{Symbol} & \textbf{Macro} & \textbf{Symbol} \\
\texttt{\textbackslash median}   & $\median$   & \texttt{\textbackslash Var}    & $\Var$ \\
\texttt{\textbackslash SD}  & $\SD$  & \texttt{\textbackslash CV}     & $\CV$ \\
\texttt{\textbackslash Bias}    & $\Bias$   & \texttt{\textbackslash AMSE}   & $\AMSE$ \\
\texttt{\textbackslash MSE} & $\MSE$  & \texttt{\textbackslash ARE}    & $\ARE$ \\
\texttt{\textbackslash AV}  & $\AV$  & \texttt{\textbackslash CRLB}   & $\CRLB$ \\
\hline
\texttt{\textbackslash TN} & $\TN$ & \texttt{\textbackslash Bern} & $\Bern$ \\
\texttt{\textbackslash Unif} & $\Unif$ & \texttt{\textbackslash Normal} & $\Normal$ \\
\texttt{\textbackslash logNormal} & $\logNormal$ & \texttt{\textbackslash Bin} & $\Bin$ \\
\texttt{\textbackslash NB} & $\NB$ & \texttt{\textbackslash HG} & $\HG$ \\
\texttt{\textbackslash Geom} & $\Geom$ & \texttt{\textbackslash Beta} & $\Beta $ \\
\texttt{\textbackslash BetaBin} & $\BetaBin$ & \texttt{\textbackslash Ga} & $\Ga$ \\
\texttt{\textbackslash Exp} & $\Exp$ & \texttt{\textbackslash Expo} & $\Expo$ \\
\texttt{\textbackslash Po} & $\Po$ & \texttt{\textbackslash Multi} & $\Multi$ \\
\texttt{\textbackslash student} & $\student$ & \texttt{\textbackslash Cauchy} & $\Cauchy$ \\
\hline
\verb|\Pareto| & $\Pareto$ & \verb|\RV| & \RV \\
\verb|\Laplace| & $\Laplace$ & \verb|\cdf| & \cdf \\
\verb|\Logistic| & $\Logistic$ & \verb|\cgf| & \cgf \\
\verb|\Dir| & $\Dir$ & \verb|\pdf| & \pdf \\
\verb|\DP| & $\DP$ & \verb|\pmf| & \pmf \\
\verb|\Inv| & $\Inv$ & \verb|\chf| & \chf \\
\verb|\F| & $\F$ & \verb|\mgf| & \mgf \\
\verb|\EF| & $\EF$ & \verb|\MLE| & \MLE \\
\verb|\NEF| & $\NEF$ & \verb|\MAP| & \MAP \\
\verb|\Med| & \Med & \verb|\MME| & \MME \\
\verb|\EB| & $\EB$ & \verb|\QME| & \QME \\
\verb|\UMVUE| & \UMVUE & \verb|\MPT| & \MPT \\
\verb|\UMPT| & \UMPT & \verb|\LRT| & \LRT \\
\verb|\mis| & \mis & \verb|\obs| & \obs \\
\verb|\com| & \com & \verb|\MCMC| & \MCMC \\
\verb|\burn| & $\burn$ & \verb|\thin| & $\thin$ \\
\verb|\ESS| & $\ESS$ & & \\
\hline
\end{tabular}
\caption{Macros and Corresponding Symbols for Statistical Notation}
\label{table2}
\end{table}

\newpage




%----------------------------------------

% citations
%\bibliographystyle{plain}
%\bibliography{citations}
%\newpage
%\appendixname
%\appendix
\end{document}

%%%----------------------------------------------------------------------------%%%
%%%----------------------------------------------------------------------------%%%
%%% 
%%% ### Packages 
%%%
%%%----------------------------------------------------------------------------%%%
%%%----------------------------------------------------------------------------%%%
\usepackage{amsmath, amsthm, amssymb, nccbbb, bm, dsfont, pifont, fontawesome, graphicx, varioref, bbold, setspace, enumitem,colortbl,mdframed,caption2,mathtools}
\usepackage{algorithm,algorithmic}
\usepackage{tikz}
\usepackage{pgfplots}
\usetikzlibrary{patterns}
\RequirePackage[round,authoryear]{natbib}
\RequirePackage[colorlinks,citecolor=red,urlcolor=red]{hyperref}
\usepackage[cal=boondox]{mathalfa}
\setlength{\topmargin}{-1in}
\setlength{\textheight}{10.5in}
\setlength{\oddsidemargin}{-.6in}
\setlength{\textwidth}{7.5in}
\hypersetup{colorlinks=true, linkcolor=myblue, citecolor=myblue, urlcolor=myblue}
\linespread{.9}
\setlist[itemize]{itemsep=0cm}
\setlist[enumerate]{itemsep=0cm}

%%%----------------------------------------------------------------------------%%%
%%%----------------------------------------------------------------------------%%%
%%% 
%%% ### Code 
%%%
%%%----------------------------------------------------------------------------%%%
%%%----------------------------------------------------------------------------%%%
\usepackage{listings}
\lstset{escapeinside=| |}
%\usepackage[usenames,dvipsnames]{color}  
\definecolor{mygray}{RGB}{242,242,242}
\definecolor{myblue}{rgb}{0.0, 0.23, 0.63}
\definecolor{myred}{rgb}{0.75, 0.0, 0.0}
\definecolor{mygreen}{rgb}{0.4, 0.69, 0.2}  
\definecolor{mypalegreen}{rgb}{0.6, 1.0, 0.6}
\definecolor{gray2}{RGB}{128,128,128} 
\definecolor{blue2}{RGB}{0,47,167}
\lstnewenvironment{R}{\lstset{ 
  language=R,
  basicstyle=\footnotesize\ttfamily, 
  numbers=left,
  numberstyle=\tiny\color{black},
  stepnumber=1,
  numbersep=5pt,
  backgroundcolor=\color{mygray},
  showspaces=false, 
  showstringspaces=false,
  showtabs=false, 
  frame=single,  
  rulecolor=\color{black},
  tabsize=4,
  captionpos=b,
  breaklines=true,
  breakatwhitespace=false,
  keywordstyle=\ttfamily\bfseries\color{myblue},
  commentstyle=\ttfamily\bfseries\color{myred},
  stringstyle=\ttfamily\bfseries\color{mygreen}
} 
}{}

\lstnewenvironment{tex}{\lstset{ 
  language=tex,
  basicstyle=\footnotesize\ttfamily, 
  numbers=left,
  numberstyle=\tiny\color{black},
  stepnumber=1,
  numbersep=5pt,
  backgroundcolor=\color{mygray},
  showspaces=false, 
  showstringspaces=false,
  showtabs=false, 
  frame=single,  
  rulecolor=\color{black},
  tabsize=4,
  captionpos=b,
  breaklines=true,
  breakatwhitespace=false,
  keywordstyle=\ttfamily\bfseries\color{myblue},
  commentstyle=\ttfamily\bfseries\color{myred},
  stringstyle=\ttfamily\bfseries\color{mygreen}
} 
}{}

%%%----------------------------------------------------------------------------%%%
%%%----------------------------------------------------------------------------%%%
%%% 
%%% ### Box
%%%
%%%----------------------------------------------------------------------------%%%
%%%----------------------------------------------------------------------------%%%

\usepackage{tcolorbox}
\tcbuselibrary{breakable}
% Custom box definition with margins migrated from md environment
\newtcolorbox{redbox}{
  colback=myred!3,      % Background color of the box
  colframe=myred,    % Border color of the box  
  leftrule=4pt,             % Thickness of the left border
  toprule=1pt,              % No top border
  bottomrule=1pt,           % No bottom border
  rightrule=1pt,            % No right border
  arc=1.2mm,                  % Rounded corners
  outer arc=1.5mm,            % Outer border radius
}


\newtcolorbox{bluebox}{
  colback=myblue!3,      % Background color of the box
  colframe=myblue,    % Border color of the box  
  leftrule=4pt,             % Thickness of the left border
  toprule=0pt,              % No top border
  bottomrule=0pt,           % No bottom border
  rightrule=0pt,            % No right border
  arc=1.2mm,                  % Rounded corners
  outer arc=1.5mm,            % Outer border radius
}

\newtcolorbox{greenbox}{
  colback=mygreen!3,      % Background color of the box
  colframe=mygreen,    % Border color of the box  
  leftrule=4pt,             % Thickness of the left border
  toprule=0pt,              % No top border
  bottomrule=0pt,           % No bottom border
  rightrule=0pt,            % No right border
  arc=1.2mm,                  % Rounded corners
  outer arc=1.5mm,            % Outer border radius
}

\newtcolorbox{graybox}{
  colback=gray2!3,      % Background color of the box
  colframe=gray2,    % Border color of the box  
  leftrule=4pt,             % Thickness of the left border
  toprule=0pt,              % No top border
  bottomrule=0pt,           % No bottom border
  rightrule=0pt,            % No right border
  arc=1.2mm,                  % Rounded corners
  outer arc=1.5mm,            % Outer border radius
}



%\newtcolorbox{mybox}{colback=yellow!5!white, colframe=gray!60!black, breakable}
\newcommand{\Solution}{\noindent{\color{myblue}{ \textsc{Solution}:~$\Big.$}}}
\newenvironment{sol}
{\Solution \par }{}


\AtBeginEnvironment{definition}{\begin{redbox} }
\AtEndEnvironment{definition}{\end{redbox}}	

\AtBeginEnvironment{proposition}{\begin{redbox} }
\AtEndEnvironment{proposition}{\end{redbox}}	

\AtBeginEnvironment{lemma}{\begin{redbox} }
\AtEndEnvironment{lemma}{\end{redbox}}	
\AtBeginEnvironment{theorem}{\begin{redbox} }
\AtEndEnvironment{theorem}{\end{redbox}}	

\AtBeginEnvironment{corollary}{\begin{redbox} }
\AtEndEnvironment{corollary}{\end{redbox}}	
\AtBeginEnvironment{condition}{\begin{redbox} }
\AtEndEnvironment{condition}{\end{redbox}}	

\AtBeginEnvironment{remark}{\begin{graybox} }
\AtEndEnvironment{remark}{\end{graybox}}	

\AtBeginEnvironment{example}{\begin{bluebox} }
\AtEndEnvironment{example}{\end{bluebox}}	

\AtBeginEnvironment{exercise}{\begin{bluebox} }
\AtEndEnvironment{exercise}{\end{bluebox}}	

\AtBeginEnvironment{question}{\begin{bluebox} }
\AtEndEnvironment{question}{\end{bluebox}}	





%%%----------------------------------------------------------------------------%%%
%%%----------------------------------------------------------------------------%%%
%%% 
%%% ### Theorem style structures 
%%%
%%%----------------------------------------------------------------------------%%%
%%%----------------------------------------------------------------------------%%%

\numberwithin{equation}{section}
\theoremstyle{plain}
\newtheorem{theorem}{Theorem}[section]
\newtheorem{lemma}[theorem]{Lemma}
\newtheorem{corollary}[theorem]{Corollary}
\newtheorem{proposition}[theorem]{Proposition}
\newtheorem{condition}{Condition}[section]
\newtheorem{definition}{Definition}[section]
\theoremstyle{definition}
\newtheorem{example}{Example}[section]
\newtheorem{exercise}{Exercise}[section]
\newtheorem{remark}{Remark}[section]
\newtheorem{question}{Question}[section]




%%%----------------------------------------------------------------------------%%%
%%%----------------------------------------------------------------------------%%%
%%% 
%%% ###Math Operators 
%%%
%%%----------------------------------------------------------------------------%%%
%%%----------------------------------------------------------------------------%%%

\newcommand{\C}{\mathbb{C}}
\newcommand{\Q}{\mathbb{Q}}
\newcommand{\Z}{\mathbb{Z}}
\newcommand{\Rn}[1]{\mathbb{R}^{#1}}
\newcommand{\N}{\mathbb{N}}
\newcommand{\borel}{\mathcal{B}}
\newcommand{\family}{\mathcal{F}}
\newcommand{\ninfo}[1]{(-\infty,#1)}
\newcommand{\ninfc}[1]{(-\infty,#1]}
\newcommand{\pinfo}[1]{(#1,+\infty)}
\newcommand{\pinfc}[1]{[#1,+\infty)}
\DeclarePairedDelimiter{\pa}{\lparen}{\rparen}
\DeclarePairedDelimiter{\br}{[}{]}
\DeclarePairedDelimiter{\cbr}{\{}{\}}
\DeclarePairedDelimiter{\oc}{\lparen}{]}%%left open right close interval
\DeclarePairedDelimiter{\co}{[}{\rparen}%%right open left close interval
\DeclarePairedDelimiter{\inner}{\langle}{\rangle}
\DeclarePairedDelimiter{\abs}{\lvert}{\rvert}
\DeclarePairedDelimiter{\norm}{\Vert}{\Vert}
\DeclarePairedDelimiter{\floor}{\lfloor}{\rfloor}
\DeclarePairedDelimiter{\ceil}{\lceil}{\rceil}
\newcommand{\p}{\partial}
\newcommand{\dd}{\textnormal{d}}
\newcommand{\dv}[3]{\frac{\dd^{#3} {#1}}{\dd {#2}^{#3}}}
\newcommand{\pdv}[3]{\frac{\p^{#3} {#1}}{\p {#2}^{#3}}}
\newcommand{\mat}[1]{\begin{matrix} #1 \end{matrix}}
\newcommand{\smat}[1]{\begin{smallmatrix} #1 \end{smallmatrix}}
\newcommand{\bmat}[1]{\begin{bmatrix} #1 \end{bmatrix}}
\newcommand{\bsmat}[1]{\begin{bsmallmatrix} #1 \end{bsmallmatrix}}
\newcommand{\pmat}[1]{\begin{pmatrix} #1 \end{pmatrix}}
\newcommand{\psmat}[1]{\begin{psmallmatrix} #1 \end{psmallmatrix}}
\DeclareMathOperator*{\argmin}{arg\,min}
\DeclareMathOperator*{\argmax}{arg\,max}
\DeclareMathOperator{\sgn}{sgn}
\newcommand{\pr}{\mathsf{P}} 
\newcommand{\E}{\mathsf{E}} 
\newcommand{\Cov}{{\mathsf{Cov}}} 
\newcommand{\Corr}{{\mathsf{Corr}}} 
\newcommand{\Var}{{\mathsf{Var}}}
\newcommand{\inD}{    \overset{ \textnormal{d}   }{\rightarrow} }
\newcommand{\inAS}{   \overset{ \textnormal{a.s.}   }{\rightarrow} }
\newcommand{\inP}{    \overset{ \textnormal{pr}    }{\rightarrow} }
\newcommand{\inLp}{   \overset{ \mathcal{L}^p }{\rightarrow} }
\newcommand{\inMSE}{  \overset{ \textnormal{qm} }{\rightarrow} }
\newcommand{\inQM}{   \overset{ \textnormal{qm} }{\rightarrow} }
\newcommand{\I}[1]{\mathbf{1}_{\{#1\}}}
\def\independenT#1#2{\mathrel{\rlap{$#1#2$}\mkern4mu{#1#2}}}
\newcommand{\indep}{\protect\mathpalette{\protect\independenT}{\perp}}
\newcommand{\iid}{\textsc{iid}} 
\newcommand{\simIID}{   \overset{ \iid   }{\sim} }
\newcommand{\simIND}{   \overset{ {\indep}   }{\sim} }
\newcommand{\diag}{\mathop{\mathrm{diag}}}
\newcommand{\T}{\mathop{\mathrm{T}}}
%%%----------------------------------------------------------------------------%%%
%%%----------------------------------------------------------------------------%%%
%%% 
%%% ###Statistical Notion
%%%
%%%----------------------------------------------------------------------------%%%
%%%----------------------------------------------------------------------------%%%
\DeclareMathOperator{\logit}{logit}
\DeclareMathOperator{\expit}{expit}
\newcommand{\median}{\mathop{\mathsf{median}}}
\newcommand{\SD}{{\mathsf{SD}}}
\newcommand{\CV}{{\mathsf{CV}}}
\newcommand{\Bias}{{\mathsf{Bias}}}
\newcommand{\AMSE}{\operatorname{\mathsf{AMSE}}}
\newcommand{\MSE}{\operatorname{\mathsf{MSE}}}
\newcommand{\ARE}{\mathsf{ARE}}
\newcommand{\AV}{\mathsf{AV}}
\newcommand{\CRLB}{{\mathsf{CRLB}}}
\newcommand{\TN}{\textnormal{TN}} 
\newcommand{\Bern}{\textnormal{Bern}} 
\newcommand{\Unif}{\textnormal{Unif}} 
\newcommand{\Normal}{\textnormal{N}} 
\newcommand{\logNormal}{\textnormal{LN}} 
\newcommand{\Bin}{\textnormal{Bin}} 
\newcommand{\NB}{\textnormal{NB}} 
\newcommand{\HG}{\textnormal{HG}} 
\newcommand{\Geom}{\textnormal{Geom}} 
\newcommand{\Beta }{\textnormal{Beta}} 
\newcommand{\BetaBin}{\textnormal{Beta-Bin}}
\newcommand{\Ga}{\textnormal{Ga}} 
\newcommand{\Exp}{\textnormal{Exp}} 
\newcommand{\Expo}{\textnormal{Expo}} 
\newcommand{\Po}{\textnormal{Po}} 
\newcommand{\Multi}{\textnormal{Multi}} 
\newcommand{\student}{\textnormal{t}} 
\newcommand{\Cauchy}{\textnormal{Cauchy}} 
\newcommand{\Pareto}{\textnormal{Pareto}} 
\newcommand{\Laplace}{\textnormal{Laplace}} 
\newcommand{\Logistic}{\textnormal{Logistic}} 
\newcommand{\Dir}{\textnormal{Dir}} 
\newcommand{\DP}{\textnormal{DP}} 
\newcommand{\Inv}{\textnormal{Inv-}} 
\newcommand{\F}{\textnormal{F}} 
\newcommand{\RV}{\textsc{rv}}
\newcommand{\cdf}{\textsc{cdf}} 
\newcommand{\cgf}{\textsc{cgf}} 
\newcommand{\pdf}{\textsc{pdf}} 
\newcommand{\pmf}{\textsc{pmf}} 
\newcommand{\chf}{\textsc{chf}} 
\newcommand{\mgf}{\textsc{mgf}}
\newcommand{\EF}{\textsc{EF}}
\newcommand{\NEF}{\textsc{NEF}}
\newcommand{\MLE}{\textsc{mle}}
\newcommand{\MAP}{\textsc{MAP}}
\newcommand{\Med}{\textsc{Med}}
\newcommand{\MME}{\textsc{mme}}
\newcommand{\EB}{\textsc{EB}}
\newcommand{\QME}{\textsc{qme}}
\newcommand{\UMVUE}{\textsc{umvue}}
\newcommand{\MPT}{\textsc{MPT}}
\newcommand{\UMPT}{\textsc{UMPT}}
\newcommand{\LRT}{\textsc{LRT}}
\newcommand{\mis}{\textsc{mis}}
\newcommand{\obs}{\textsc{obs}}
\newcommand{\com}{\textsc{com}}
\newcommand{\MCMC}{\textsc{MCMC}}
\DeclareMathOperator{\burn}{burn}
\DeclareMathOperator{\thin}{thin}
\DeclareMathOperator{\ESS}{ESS}





\newcommand{\lva}{{\color{myred}\ding{73}\ding{73}\ding{73}}}
\newcommand{\lvb}{{\color{myred}\ding{72}\ding{73}\ding{73}}}
\newcommand{\lvc}{{\color{myred}\ding{72}\ding{72}\ding{73}}}
\newcommand{\lvd}{{\color{myred}\ding{72}\ding{72}\ding{72}}}
\newcommand{\optional}{\noindent{\color{myblue}\faScissors}}
\newcommand{\take}{\noindent{\color{myblue}\faPaperPlaneO~\underline{\bf Takeaway}:~$\Big.$}}
\newcommand{\experiment}{\noindent{\color{myblue}\faCogs~\underline{\bf Experiment}:~$\Big.$}}
\newcommand{\intuition}{\noindent{\color{myblue}\faLightbulbO~\underline{\bf Intuition}:~$\Big.$}}

% widecheck 
\DeclareFontFamily{U}{mathx}{\hyphenchar\font45}
\DeclareFontShape{U}{mathx}{m}{n}{
      <5> <6> <7> <8> <9> <10>
      <10.95> <12> <14.4> <17.28> <20.74> <24.88>
      mathx10
      }{}
\DeclareSymbolFont{mathx}{U}{mathx}{m}{n}
\DeclareFontSubstitution{U}{mathx}{m}{n}
\DeclareMathAccent{\widecheck}{0}{mathx}{"71}
\DeclareMathAccent{\wideparen}{0}{mathx}{"75}

\def\cs#1{\texttt{\char`\\#1}}






